\chapter*{Заключение}
\label{cha:Conclusion}

%Заключение последовательное логически стройное изложение итогов исследования в соответствии с целью и задачами, поставленными и сформулированными во введении. В нем содержатся выводы и определяются дальнейшие перспективы работы.

% \begin{itemize}
%     \item В ходе работы определена целесообразность использования метода конечного объема для моделирования аттракторов внутренних волн.
%     \item Найдены теоретические диапазоны частот колебаний волнопродуктора, которые способны порождать аттракторы.
%     \item Установлено, что при воздействии на резервуар со стратифицированной жидкостью волнопродуктором, который совершает колебания описываемые суммой двух монохроматических функций образуется суперпозиция двух аттракторов в одном резервуаре.
%     \item Обнаружено, что при близости частот двух монохроматических функций возбуждающих внутренние волны в резервуаре образуются амплитудные биения. 
% \end{itemize}



Показано, что результаты моделирования аттракторов внутренних волн, полученные с помощью методов конечного объёма при увеличении количества ячеек стремятся к результатам, полученным с помощью метода высокого порядка. Таким образом сделан вывод о целесообразности дальнейшего использования конечно объёмной реализации квазигидродинамических уравнений для моделирования аттракторов внутренних волн. 
Аналитически определены  границы частотного диапазона существования аттрактора. Выведены формулы расчёта каждой из границ в зависимости от геометрических характеристик резервуара и частоты плавучести.


Получены результаты моделирования бигармонических аттракторов, то есть таких, которые возникают при воздействии на жидкость в трапециевидном резервуаре с двумя частотами, попадающими в интервал существования аттрактора. Установлено, что в этом случае картина течения каждой частоты по отдельности накладывается друг на друга. В резервуаре появляются два независимых аттрактора, каждый из которых совершает движение с собственной частотой, а взаимодействуют они только в точках пересечения. 


Рассмотрены различные комбинации частот из диапазона существования аттракторов. Когда частоты совпадают, это фактически удваивает амплитуду колебаний волнопродуктора монохроматического аттрактора. В случае большой амплитуды колебаний волнопродуктора аттрактор начинает поражать дочерние волны и насыщает спектр. В случае разнесённых частот аттракторы практически не взаимодействуют, амплитуды не складываются. В случае, когда частоты приближены друг к другу, в момент совпадения фаз наблюдается взаимодействие аттракторов, тогда постепенно спектр частот начинает насыщаться, но в момент разности фаз спектр возвращается в исходное состояние. В случае, когда частоты располагаются еще ближе друг к другу, на частотно-временной диаграмме наблюдается еще более активное взаимодействие аттракторов, а на графике зависимости средней кинетической энергии в резервуаре от времени наблюдаются биения.


Выяснено, что с большой точностью сумма средних кинетических энергий аттракторов, образующихся при монохроматическом режиме колебаний волнопродуктора, равна средней кинетической энергии бигармонического аттрактора.

Работа представляет собой первый шаг к моделированию аттракторов как природного явления в океане. Для этого необходимо разработать инструменты численного моделирования монохроматического аттрактора в условиях геометрии, приближенной к реальной. А также исследовать течения возникающие при воздействии на стратифицированную жидкость суммой нескольких монохроматических колебаний. 

Для реализации инструмента численного моделирования в сложной геометрии была разработана программа, которая подлежала государственной регистрации номер 2018663951. Разработанный инструмент имеет ряд преимуществ относительно уже существующих программных средств, такие как точность, гибкость, возможность встроить дополнительные модули физических процессов и возможность работать со сложной геометрией на неортогональных сетках. Количественное соответствие результатов моделирования методом конечного объема и методом спектральных элементов показывают целесообразность дальнейшего развития метода конечного объема на базе квазигидродинамических уравнений. Соответствие предсказанной трассировкой лучей формы бигармонического аттрактора и результатов моделирования с помощью регуляризованных уравнений дает возможность сделать заключение о целесообразности дальнейшего применения. 

Количественное исследование показало, что после выхода системы на режим установившихся колебаний средняя кинетическая энергия системы, возбуждаемой бигармоническим возмущением, с высокой точностью равна сумме энергий аттракторов, возбуждаемых монохроматическими возмущениями по отдельности. Таким образом, в линейном режиме с высокой точностью соблюдается принцип линейной суперпозиции, что выполняется также при малой разности частот. 

В нелинейном случае средняя энергия системы существенным образом отличается от суммы энергий составляющих. Наблюдается режим <<биений>> сопровождающийся <<вспышками>> волновой турбулентности, возникающей вследствие каскада триадных взаимодействий. При этом уровень пульсаций кинетической
энергии на фазе роста огибающей амплитуды волнопродуктора, может на порядок превышать уровень, соответствующий спаду амплитуды колебаний волнопродуктора.

Реализован квазигидродинамический подход на базе конечно объёмного пакета OpenFOAM. Программа охватывает дозвуковой и трансзвуковой диапазон скоростей, позволяет проводить численное моделирование вязких течений с переносом. Исходный код, тестовые примеры и документация размещена в открытом хранилище исходного кода на github. 