\Introduction

Внутренние волны возникают в стратифицированных жидкостях между слоями различной плотности. Самым распространенным примером стратифицированной жидкости является океан.  Внутренние волны распространяются согласно дисперсионному закону \cite{MowbrayRarity1967}, которое связывает частоту волн и угол наклона волнового пучка по отношению к вектору силы тяжести, но не содержит масштаба длины. 

Подчиняясь дисперсионному соотношению внутренние волны при отражении от наклонных поверхностях могут фокусироваться. Под фокусировкой подразумевается увеличивающуюся амплитуду колебаний стратифицированной жидкости. При определенных геометрически параметрах морского дна или резервуара со стратифицированной жидкостью после многократных отражений внутренние волны начинают циркулировать по замкнутой траектории. На этой траектории наблюдается многократное увеличение амплитуды колебаний, а сама траектория называется аттрактором внутренних волн.

Аттракторы в океанах оказывают влияние на процессы перемешивания, перераспределение кинетической энергии между течениями различных масштабов, осаждения примесей, динамику спускаемых аппаратов и миграцию живых организмов. Это обуславливает \textbf{актуальность} изучения явления аттракторов внутренних гравитационных волн.

Важной задачей изучения аттракторов внутренних волн с помощью численных методов является обеспечение возможности проводить численные эксперименты с геометрией, приближенной к геометрии реального дна океана. Выполнение этой задачи ускорило и удешевило бы процесс непосредственного поиска аттракторов внутренних волн в океане, и изучение влияния аттркаторов на турбулентные режимы течения в водоемах. Метод спектральных элементов, который обеспечивает достаточную точность воспроизведения результатов эксперимента, ограничен в своей реализации сложностью геометрии расчетной области. В свою очередь, метод конечного объема позволяет работать со сложной геометрией, которая способна имитировать поверхность океанического дна, но стандартные реализации не обладают достаточной точностью для количественного воспроизведения эксперимента. Кроме того, монохроматический источника возмущений может не описывать реальные внешние воздействия. Зачастую, при моделировании явлений, связанных с образованием аттракторов в реальных условиях, необходимо учитывать несколько приливных воздействий \cite{Garrett1972} и изменение стратификации.

 

\paragraph{Цель работы} -- изучение явления бигармонического аттрактора, которое возникает при воздействии на стратифицированную жидкость двухчастотным волнопродуктором.  
С этой целью были поставлены следующие задачи \textbf{задачи}:

\begin{itemize}

  \item Нахождение интервала частот внешних воздействий, при которых возникает аттрактор внутренних волн.
  
  % Изучение интервалов частот внешних воздействий и других параметров, при которых происходит аккумуляция волновой энергии, в частности волновых аттракторов.

%    \item Нахождение частотных параметров приводящих к образованию аттракторов в резервуаре.
    
  % \item Обзор существующих методов моделирования аттракторов внутренних волн. Выявление их достоинств и недостатков.
    
  \item Реализация численных экспериментов с помощью двух подходов: спектрально-элементного и конечно-объемного.

  \item Разработка новой программы для моделирования аттракторов внутренних волн на основе квазигидродинамического подхода.
    
  \item Верификация результатов численного моделирования.

  \item Описание особенностей волновых режимов при бигармоническом воздействии и значительно отличающихся частотах воздействия и малых амплитудах.

  \item Описание особенностей волновых режимов при бигармоническом воздействии, близких частотах воздействия и малых амплитудах.
    
  \item Описание особенностей нелинейных волновых режимов при бигармоническом воздействии и близких частотах воздействия.

  \item Сравнение динамики средней кинетической энергии и пульсации кинетической энергии для монохроматического режима и различных бигармонических режимов.
    

    
\end{itemize}

\paragraph{Методы решения поставленных задач}

Для решения поставленных задач были использованы методы математического моделирования механики сплошных сред, такие как метод спектральных элементов и метод конечного объема. Для предсказания формы аттрактора внутренних волн использовался метод трассировки лучей. Для анализа данных использовался метод построения частотно-временных диаграмм при помощи быстрого преобразования Фурье.

\paragraph{Научная новизна работы} выражается в конкретных результатах:
\begin{enumerate}[1.]
  \item Получены аналитические выражения для границ частотного интервала существования аттракторов внутренних волн.% конфигурации (1,1). 
    
  \item Получена геометрия течения, которая возникает в трапециевидном резервуаре, наполненном стратифицированной жидкостью при воздействии на жидкость внешними возмущениями с двумя различными частотами. 
    
  \item Проведён анализ результатов моделирования аттрактора внутренних волн при бигармоническом воздействии, полученных с помощью метода спектральных элементов. Для различных комбинаций возмущающих частот построен спектр, частотно-временная диаграмма и зависимость средней кинетической энергии от времени. 
    
  \item Реализован квазигидродинамический подход на базе метода конечного элемента. Проведено сопоставление результатов моделирования методов конечных объемов и методом спектральных элементов.
\end{enumerate}

\paragraph{Достоверность результатов}

Достоверность полученных результатов гарантируется строгой математической постановкой, верификацией и валидацией разработанного алгоритма для решения поставленной задачи.

% \paragraph{Объектом исследования} являются волновые режимы возникающие %в естественных условиях 
% при двух источниках внешних воздействий на стратифицированную жидкость в трапециевидном резервуаре.

% %Приложения включают в себя задачи океанологии, астрофизики и технических вращающихся систем при периодических воздействиях. 

% %\paragraph{В исследовании использованы следующие методы:}
% \begin{itemize}
%   \item [    В исследовании использованы \textbf{
% методы:}]
%   \item методы численного моделирования конечного объема;
%   \item метод спектральных элементов;
%   \item метод трассировки лучей;
%   \item Фурье анализ полученных результатов, в том числе по скользящему окну;
%   \item разложение по эмпирическим модам;
% \end{itemize}

% % В работе рассматриваются резервуары в форме трапеций различных конфигураций, заполненных стратифицированной жидкостью. Одна из стенок резервуара представляет собой волнопродуктор, который порождает внутренние волны в стратифицированной среде. Результаты разработки предоставляют возможность проводить моделирование аттракторов внутренних волн в условиях сложной геометрии и неортогональных сетках. 



\paragraph{Практическая значимость} 

Ранее эксперименты по исследованию бигармонических аттракторов, как численные так и натурные, не проводились. Теоретически, бигармонический аттрактор представляет собой новую устойчивую структуру, которая образуется в стратифицированной жидкости при воздействии на нее периодическим двухчастотным возмущением.

Положения и выводы диссертационного исследования могут быть использованы для подбора параметров  волнового аттрактора в лабораторных условиях или при численном моделировании. Среди возможных приложений результатов работы — задачи моделирования аттракторов внутренних волн на сложных геометриях, задачи моделирования течений со сложным спектром частотных воздействий на стратифицированную жидкость. Работа является первым шагом к моделированию течений, возникающих в условиях, приближенных к реальным океаническим, что позволит выяснить форму и вид природных аттракторов внутренних волн. Комбинация методов конечного объёма и квазигидродинамических уравнений позволила добиться существенного улучшения в точности моделирования и дала инструмент к  усложнению геометрии расчётной области. Разработанная программа может быть применена не только к задачам моделирования аттрактора, но и к другим задачам гидродинамики с дозвуковыми и трансзвуковыми скоростями.

\paragraph{На защиту выносятся следующие положения:}
\begin{itemize}

  %\item Найдены аналитические выражения для границ диапазонов частот колебаний волнопродуктора, которые способны порождать аттракторы.

  \item Показано, что при значительном отличии частот внешних воздействий и малых амплитудах воздействий волновой режим представляет собой совокупность независимо существующих волновых аттракторов.

  \item Показано, что при близких частотах внешних воздействий и малых амплитудах возникает режим с биениями, характерной особенностью которых является малая амплитуда пульсаций на убывающем склоне огибающей.

  \item Показано, что при близких частотах внешних воздействий и средних амплитудах возникают биения, на одном цикле которых успевает происходить переход к турбулентности через триадные резонансы, и реламинаризация.
    
  \item Обнаружено наличие фазового сдвига между биениями на волнопродукторе и биениями средней кинетической энергии во всем объеме.
    
%   \item 
%     Сравнение динамики средней кинетической энергии и пульсации кинетической энергии для монохроматического режима и различных бигармонических режимов указывает на то, что на убывающем склоне огибающей большая доля кинетической энергии переходит в бегущие волны по сравнению с возрастающей фазой.

  \item Разработана и верифицирована новая программа для моделирования аттракторов внутренних волн и в целом динамики стратифицированных сред.
    
%  \item Реализация численных экспериментов с помощью двух подходов: спектрально-элементного и конечно-объемного.

%  \item Проведена верификация результатов численного моделирования.
%    Проведение численных экспериментов различными методами.
    
%  \item Количественный анализ результатов численных экспериментов. 
    
%  \item Верификация разработанной программы.
    
\end{itemize}

\paragraph{Личный вклад автора}

Исследования, результаты которых выносятся на защиту, были получены лично соискателем. Соискатель аналитически нашел диапазон частот внешнего воздействия при которых образуется аттрактор внутренних волн. Соискатель подобрал параметры эксперимента, провел расчеты и проанализировал полученные данные. Также принимал непосредственное участие в разработке реализации квазигидродинамического подхода на базе открытого программного комплекса OpenFOAM. Научный руководитель И. Н. Сибгатуллин поставил первоначальную задачу и участвовал в обсуждении результатов. 

\paragraph{Апробация работы}

Материалы диссертации представлялись на различных конференциях, семинарах, как российских так и международных:


\begin{itemize}
  \item Открытая международная конференция ИСП РАН им. В.П.Иванникова. 5-6 декабря 2019 г, г. Москва Главное здание Российской академии наук (устный доклад).
  \item Международная конференция «Суперкомпьютерные технологии математического моделирования» (СКТеММ’19), 19-21 июня 2019, г. Москва (устный доклад).
  \item 13th OpenFOAM Workshop, Shanghai, China, Китай, 24-29 июня 2018 (устный доклад).
  \item XXIII международная конференция «Нелинейные задачи теории гидродинамической устойчивости и турбулентность». 25 февраля - 4 марта 2018, Московская область, г. Звенигород (стендовый доклад).
  \item Рязанов Д.А. Открытая конференция ИСП РАН им. В.П. Иванникова. 30 ноября - 1 декабря 2017 г. Москва главное здание Российской академии наук (стендовый доклад).
\end{itemize}

\paragraph{Публикации}

По результатам диссертации опубликовано 12 научных работ, входящих в базы данных и системы цитирования РИНЦ, Scopus, Web of Science, 2 из них входят в Перечень рецензируемых научных изданий, рекомендованных Высшей аттестационной комиссией. Зарегистрирована программа для ЭВМ. Работа поддержана российского научного фонда номер 19-11-00169. 

\paragraph{Структура и объем диссертации}

Диссертация состоит из введения, обзора литературы, трех глав, заключения и списка литературы. Текст работы содержит 106 печатных страниц, 59 рисунков и 6 таблиц. Список литературы включает в себя 90 наименований. 

% В ходе работ был разработан программный продукт, который подлежал государственной регистрации № 2018663951.
