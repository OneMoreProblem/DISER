% Преамбула TeX-файла

% 1. Стиль и язык
\documentclass[utf8x]{G7-32} % Стиль (по умолчанию будет 14pt)
\usepackage[T2A]{fontenc}
\usepackage[russian]{babel}
\usepackage{setspace,lipsum}
\PassOptionsToPackage{shorthands=off}{babel}
\usepackage{bm}

\usepackage{mathtools}
\usepackage{caption}
\usepackage{subcaption}
\usepackage{setspace}
\usepackage{pgfplots}
\pgfplotsset{compat=newest}
\usepackage{filecontents}
\usepackage{animate}
\usepackage{graphics}
\usepackage{color}
\usepackage{fancyhdr}
\usepackage{lipsum}

\newcommand\blfootnote[1]{%
  \begingroup
  \renewcommand\thefootnote{}\footnote{#1}%
  \addtocounter{footnote}{-1}%
  \endgroup
}
\def\alert#1{\textcolor{red}{#1}}

\usepgfplotslibrary{dateplot}


% Остальные стандартные настройки убраны в preamble.inc.tex.
\include{preamble.inc}

% Настройки листингов.
\include{listings.inc}

% Полезные макросы листингов.
\include{macros.inc}


\begin{document}

\thispagestyle{empty}
\pagestyle{fancy}
\fancyhf{}
\fancyhead[C]{\thepage}
\renewcommand{\headrulewidth}{0pt}
\newgeometry{top=2cm,bottom=3cm,left=3cm,right=1cm}
{
\singlespacing
\begin{center}

ФГБУН <<Институт системного программирования им. В.П. Иванникова>>
\medskip
\hrule
\medskip
\end{center}

\vspace{20mm}
\begin{flushright}
На правах рукописи

%{\sl УДК 519.713.3}
\end{flushright}

\vspace{25mm}
\begin{center}
{\large Рязанов Даниил Александрович}
\end{center}

\vspace{5mm}
\begin{center}
{\bf \large Бигармонические аттракторы внутренних волн 
\par}

\vspace{10mm}
{
01.02.05~--- Механика жидкости газа и плазмы
}

%\parbox{0.88\textwidth}{
%Специальность 05.13.17~---
%Теоретические основы информатики
%}
%\parbox{0.88\textwidth}{
%\vspace{5mm}
%Специальность 05.13.19~---
%Методы и системы защиты информации,
%}
%информационная безопасность

%\begin{tabular}{r c l}
%Специальность 05.13.17& --- &Теоретические основы информатики\\[3mm]
%Специальность 05.13.19& --- &\parbox[t]{10cm}{Методы и системы защиты информации,\\ \centering информационная безопасность}
%\end{tabular}



\vspace{10mm}
Автореферат диссертации на соискание ученой степени кандидата физико-математических наук
\end{center}

% \vspace{16mm}
% \begin{flushright}
% Научный руководитель:\\[2mm]
% д.ф.-м.н., \\
% Веденеев~В.\,В.\\

% \end{flushright}

\vfill
\begin{center}
{Москва -- 2021}
\end{center}
}
\newpage
\restoregeometry

\newpage
\begin{spacing}{1.2}
\pagestyle{empty}
\begin{flushleft}
  Работа выполнена в ИСП им. В.П. Иванникова РАН
\end{flushleft}

\begin{flushleft}
\begin{tabular}{p{5.8cm} p{10.5cm}}
    Научный руководитель: & к. ф.-м. н. Сибгатуллин Ильяс Наилевич старший научный сотрудник, института океанологии РАН  \\

    \\

    Официальные оппоненты: & д.т.н., доцент Петров Петр Петрович, профессор кафедры математического обеспечения ЭВМ факультета вычислительной математики и кибернетики Нижегородского государственного университета им. Н.И.Лобачевского (г. Нижний Новгород)\\

    \\

    & д.т.н., доцент Петров Петр Петрович, профессор кафедры математического обеспечения ЭВМ факультета вычислительной математики и кибернетики Нижегородского государственного университета им. Н.И.Лобачевского (г. Нижний Новгород)
    
    \\

    Ведущая организация: & Институт вычислительной математики и математической геофизики СО РАН (г. Новосибирск)
    
\end{tabular}
\end{flushleft}

\begin{flushleft}
    Защита состоится 26 апреля 2021 г. в 11:00 часов на заседании диссертационного совета Д 002.024.01, созданного на базе ИПМ им. М.В. Келдыша РАН, 125047, Москва, Миусская пл., д.4

    \setlength{\parskip}{1em}

    С диссертацией можно ознакомиться в библиотеке и на сайте ИПМ им. М.В. Келдыша РАН http://keldysh.ru 

    \setlength{\parskip}{1em}

    Автореферат разослан «\underline{\hspace{1cm}}» \underline{\hspace{4cm}} 2021 г.    
\end{flushleft}

\setlength{\parskip}{1em}

\begin{flushleft}
    \begin{tabular}{p{10cm} p{5cm}}

        Учёный секретарь диссертационного совета \\ к.ф.-м.н.  
    &  М.Г. Широбоков 
        
    \end{tabular}
\end{flushleft}
\end{spacing}
\newpage
\pagestyle{fancy}
\setcounter{page}{3}

\setcounter{secnumdepth}{-1}

\section{Общая характеристика работы}

% Важной задачей изучения аттракторов внутренних волн с помощью численных методов является обеспечение возможности проводить численные эксперименты с геометрией, приближенной к геометрии реального дна океана. Выполнение этой задачи ускорило и удешевило бы процесс непостредственного поиска аттракторов внутренних волн в океане, и изучение влияния аттркаторов на турбулентные режимы. Метод спектральных элементов, который обеспечивает достаточную точность воспроизведения результатов эксперимента, ограничен в своей реализации сложностью геометрии расчетной области. В свою очередь, метод конечного объема позволяет работать со сложной геометрией, которая способна имитировать поверхность океанического дна, но стандартные реализации не обладают достаточной точностью. Кроме того, монохроматический источника возмущений может не описывать реальные внешние воздействия. Зачастую, при моделировании явлений, связанных с образованием аттракторов в реальных условиях, необходимо учитывать несколько приливных воздействий и изменение стратификации.

\paragraph{Актуальность работы.}

Внутренние волны из-за особенностей своего распространения имеют возможность фокусироваться. Многократное отрежение от наклонных поверхностей, которые явялются частью рельефа морского дна, ведет к накоплению кинетической энергии и интенсификации движения стратифицированной жидкости. Такое явления называется волновым аттрактором. Волновые аттракторы в океанах из-за значительных запасов кинетической энергии должны влиять на необратимое перемешивание стратифицированной жидкости, седиментацию примесей, поведение живых организмов. 

Волновые структуры, называемые волновыми аттракторами, в явном виде воспроизводятся на лабораторных установках, но их моделирование в условиях приближенных к природным имеет ряд значительных сложностей. Прежде всего из-за перехода к турбулентному режиму, большого количества источников внешних возмущений(вместо одного при лабораторных условиях), сложной геометрии морского дна и нелинейный профиль солености. Эти особенности течения в условиях приближенных к реальным может исказить четкую структуру волнового аттрактора, наблюдаемую в лабораторных условиях. 

В этой работе рассматривается периодичекое воздействие на стратифицированную жидкость с двумя частотами, которые соотвествуют двум различным конфигурациям волновых аттракторов. Выясняется возможность существования аттракторов при бигармоническом воздействии на стратифицированную жидкость и способы взаимодействия внутренних волн различных частот. 

Решение задач моделирования аттракторов внутренних волн в условиях сложных геометриях, порождаемых источниками возмущения различной частоты и амплитуды поможет описать течения, возникающие в результате многократных отражений внутренних волн от рельефа дна океана. Результаты моделирования позволят дать первичную оценку влияния аттракторов на перемешивание в стратифицированной жидкости, на процессы седиментации различных веществ, на эрозию конструкций и рельефа в областях повышенной интенсивности движения жидкости, на паттерны поведения различных форм жизни в условиях сосуществования с аттрактором внутренних волн. Результаты работы представляют собой интерес для приложений в океанологии, экологии, биологии, астрофизики и вращающихся технических систем. 

\paragraph{Цель работы} -- изучение свойств волновых режимов, возникающих при бигармонических воздействиях на стратифицированную жидкость в области с наклонной границей по отношению к силе тяжести.
Для этого поставлены следующие \textbf{задачи}:

\begin{itemize}

  \item Изучение интервалов частот внешних воздействий и других параметров, при которых происходит аккумуляция волновой энергии, в частности волновых аттракторов.

%    \item Нахождение частотных параметров приводящих к образованию аттракторов в резервуаре.
    
  \item Обзор существующих методов моделирования аттракторов внутренних волн. Выявление их достоинств и недостатков.
    
  \item Реализация численных экспериментов с помощью двух подходов: спектрально-элементного и конечно-объемного.

  \item Разработка новой программы для моделирования аттракторов внутренних волн на основе анализа существующих подходов.
    
  \item Верификация результатов численного моделирования.

  \item Описание особенностей волновых режимов при бигармоническом воздействии и значительно отличающихся частотах воздействия и малых амплитудах.

  \item Описание особенностей волновых режимов при бигармоническом воздействии, близких частотах воздействия и малых амплитудах.
    
  \item Описание особенностей нелинейных волновых режимов при бигармоническом воздействии и близких частотах воздействия.

  \item Сравнение динамики средней кинетической энергии и пульсации кинетической энергии для монохроматического режима и различных бигармонических режимов.
    
%    Проведение численных экспериментов различными методами.
    
%  \item Количественный анализ результатов численных экспериментов. 
    
%  \item Верификация разработанной программы.
    
\end{itemize}

\paragraph{Объектом исследования} являются волновые режимы возникающие 
%в естественных условиях 
при двух источниках внешних воздействий на стратифицированную жидкость в трапециевидном резервуаре.

%Приложения включают в себя задачи океанологии, астрофизики и технических вращающихся систем при периодических воздействиях. 

%\paragraph{В исследовании использованы следующие методы:}
\begin{itemize}
  \item [    В исследовании использованы \textbf{
методы:}]
  \item методы численного моделирования конечного объема;
  \item метод спектральных элементов;
  \item метод трассировки лучей;
  \item Фурье анализ полученных результатов, в том числе по скользящему окну;
  \item разложение по эмпирическим модам;
\end{itemize}

% В работе рассматриваются резервуары в форме трапеций различных конфигураций, заполненных стратифицированной жидкостью. Одна из стенок резервуара представляет собой волнопродуктор, который порождает внутренние волны в стратифицированной среде. Результаты разработки предоставляют возможность проводить моделирование аттракторов внутренних волн в условиях сложной геометрии и неортогональных сетках. 

\paragraph{Научная новизна работы} выражается в конкретных реузьтатах:
\begin{enumerate}[1.]
  \item Получены аналитические выражения для границ частотного интервала существования аттракторов внутренних волн.% конфигурации (1,1). 
    
  \item Получена геометрия течения, которая возникает в трапециевидном резервуаре, наполненном стратифицированной жидкостью при воздействии на жидкость внешними возмущениями с двумя различными частотами. 
    
  \item Проведён анализ результатов моделирования аттрактора внутренних волн при бигармоническом воздействии, полученных с помощью метода спектральных элементов. Для различных комбинаций возмущающих частот построен спектр, частотно-временная диаграмма и зависимость средней кинетической энергии от времени. 
    
  \item Реализован квазигидродинамический подход на базе метода конечного элемента. Проведено сопоставление результатов моделирования методов конечных объемов и методом спектральных элементов.
\end{enumerate}

\paragraph{Научная и практическая значимость} 

Ранее эксперименты по исследованию бигармонических аттракторов, как численные так и натурные, не проводились. Теоретически, бигармонический аттрактор представляет собой новую устойчивую структуру, которая образуется в стратифицированной жидкости при воздействии на нее периодическим двухчастотным возмущением.

Положения и выводы диссертационного исследования могут быть использованы для подбора параметров  волнового аттрактора в лабораторных условиях или при численном моделировании. Среди возможных приложений результатов работы — задачи моделирования аттракторов внутренних волн на сложных геометриях, задачи моделирования течений со сложным спектром частотных воздействий на стратифицированную жидкость. Работа является первым шагом к моделированию течений, возникающих в условиях, приближенных к реальным океаническим, что позволит выяснить форму и вид природных аттракторов внутренних волн. Комбинация методов конечного объёма и квазигидродинамических уравнений позволила добиться существенного улучшения в точности моделирования и дала инструмент к  усложнению геометрии расчётной области. Разработанная программа может быть применена не только к задачам моделирования аттрактора, но и к другим задачам гидродинамики с дозвуковыми и трансзвуковыми скоростями.

\paragraph{На защиту выносятся следующие положения:}
\begin{itemize}

  \item Найдены аналитические выражения для границ диапазонов частот колебаний волнопродуктора, которые способны порождать аттракторы.

  \item Показано, что при значительном отличии частот внешних воздействий и малых амплитудах воздействий волновой режим представляет из себя совокупность независимо существующих волновых аттракторов.

  \item Показано, что при близких частотах внешних воздействий и малых амплитудах возникает режим с биениями, характерной особенностью которых является малая амплитуда пульсаций на убывающем склоне огибающей.

  \item Показано, что при близких частотах внешних воздействий и средних амплитудах возникают биения, на одном цикле которых успевает происходит переход к турбулентности через триадные резонансы, и реламинаризация.
    
  \item Обнаружено наличие фазового сдвига между биениями на волнопродукторе и биениями средней кинетической энергии во всем объеме.
    
%   \item 
%     Сравнение динамики средней кинетической энергии и пульсации кинетической энергии для монохроматического режима и различных бигармонических режимов указывает на то, что на убывающем склоне огибающей большая доля кинетической энергии переходит в бегущие волны по сравнению с возрастающей фазой.

  \item Разработана и верифицирована новая программа для моделирования аттракторов внутренних волн и в целом динамики стратифицированных сред.
    
%  \item Реализация численных экспериментов с помощью двух подходов: спектрально-элементного и конечно-объемного.

%  \item Проведена верификация результатов численного моделирования.
%    Проведение численных экспериментов различными методами.
    
%  \item Количественный анализ результатов численных экспериментов. 
    
%  \item Верификация разработанной программы.
    
\end{itemize}

\section{Содержание работы}

Работа состоит из введения где проведен обзор литературы, рассмотренны математические модели для изучения гравитационных и инерционных волн. Также рассматривается линеаризованная теория внутренних гравитационных волн и исследование волновых течений с помощью трассировки лучей.

Раздел с численным моделированием аттракторов внутренних гравитационных волн содержет в себе результаты моделирования аттракторов с помощью метода спектральных элементов и контрольного объе ма алгоритмами PISO и QHD на базе открытого программного продукта OpenFOAM. Также в этом разделе обусловлен выбор частот для моделирования бигармонических колебаний, приведены результаты моделирования и анализ данных.

Наконец приведен заключитильный раздел с оновными выводами и список использованных источников. 


%\begin{itemize}
%
%  \item Найдены аналитические выражения для границ диапазонов частот колебаний волнопродуктора, которые способны порождать аттракторы формы (1,1).
%    
%  \item Установлено, что при воздействии на резервуар со стратифицированной жидкостью волнопродуктором, который совершает колебания описываемые суммой двух монохроматических функций образуется суперпозиция двух аттракторов в одном резервуаре.
%    
%  \item Обнаружено, что при близости частот двух монохроматических функций возбуждающих внутренние волны в резервуаре образуются амплитудные биения. 
%    
%  \item В ходе работы определена целесообразность использования метода конечного объема для моделирования аттракторов внутренних волн.
%    
%\end{itemize}

\paragraph{Аппробация} работы была проведена на следующих конференциях:

\begin{itemize}

    \item 2019 Регуляризированные уравнения для моделирования бигармонических аттракторов внутренних волн. (Стендовый) Авторы: Даниил Рязанов, Ильяс Сибгатуллин Открытая конференция ИСП РАН им. В.П.Иванникова, Главное здание Российской академии наук, Москва, Россия, 5-6 декабря 2019.
    
    \item 2019 Моделирование бигармонических аттракторов внутренних гравитационных волн в трапециевидном резервуаре. (Устный) Автор: Рязанов Даниил. IV Международная конференция «Суперкомпьютерные технологии математического моделирования» (СКТеММ’19), г. Москва, Россия, 19-21 июня 2019
    
    \item 2018 New OpenFOAM solvers for transonic and incompressible flow simulations. (Устный) Автор: Ryazanov D.A. 13th OpenFOAM Workshop, Address: 800 Dongchuan Road, Minhang District, Shanghai, China, Китай, 24-29 июня 2018.
    
    \item 2018 БИГАРМОНИЧЕСКИЕ АТТРАКТОРЫ ВНУТРЕННИХ ВОЛН (Стендовый) Авторы: Рязанов Д.А., Ерманюк Е.В., Сибгатуллин И.Н. XXIII международная конференция «Нелинейные задачи теории гидродинамической устойчивости и турбулентность», Московская область, г. Звенигород, Россия, 25 февраля - 4 марта 2018
    
    \item 2017 Бигармонические аттракторы внутренних волн (Стендовый) Авторы: Сибгатуллин И.Н., Ерманюк Е.В., Рязанов Д.А. Открытая конференция ИСП РАН им. В.П. Иванникова 2017, Москва, Россия, 30 ноября - 1 декабря 2017
    
\end{itemize}

Результаты работы были опубликованы в сборниках тезисов конференций и периодических изданиях, входящих в РИНЦ, scopus и web of science:

\begin{itemize}

    \item Numerical simulation of three-dimensional wave attractors / I. N. Sibgatullin, E. V. Ermanyuk, K. A. Vatutin, D. A. Ryazanov, Xiulin Xu,  // The XXVII workshop of the Council of nonlinear dynamics of the Russian Academy of Sciences. — 2019. — Vol. 47, no. 1. — P. 112–115.
    
    \item Numerical simulation of internal wave attractors in horizontally elongated domains with sloping boundaries / I. Sibgatullin, X. Xiulin, E. Ermanuyk et al. // GISTAM 2019 5th International Conference on Geographical Information Systems Theory, Applications and Management. — SCITEPRESS Heraklion, Crete, Greece, 2019. — P. 366–370.
    
    \item Openfoam solver based on regularized hydrodynamic equations for high performance computing / M. V. Shatskiy, D. A. Ryazanov, K. A. Vatutin et al. // 2019 Ivannikov Memorial Workshop (IVMEM) / Ed. by С. П. Прохоров. — IEEE, 2019.
    
    \item Numerical simulation of compressible gas flows using regularized gas dynamic equations solver qgdfoam / M. V. Kraposhin, T. G. Elizarova, M. A. Istomina et al. // AIP Conference Proceedings 1959, 030013 (2018). — Author(s), 2018.
    
    \item Openfoam high performance computing solver for simulation of internal wave attractors in stratified flows using regularized hydrodynamic equations / M. Kraposhin, D. Ryazanov, T. Elizarova et al. // Proceedings of the 2018 Ivannikov ISPRAS Open Conference (ISPRAS, 22-23 Nov. 2018). — IEEE Xplore Digital Library. — United States: United States, 2018.
    
    \item Development of openfoam solver for compressible viscous flows simulation using quasi-gas dynamic equations / M. V. Kraposhin, D. A. Ryazanov, E. V. Smirnova et al. // 2017 Ivannikov ISPRAS Open Conference (ISPRAS). — Vol. 1. — United States: United States, 2017.
    
\end{itemize}





%Разместить в самом низу страницы
\blfootnote{
\begin{centering}
    Подписано в печать 12.01.2020. Формат 60х84/16.  Усл. печ. л. 0,9. Тираж 60 экз. Заказ А-3. ИПМ им.М.В.Келдыша РАН. 125047, Москва, Миусская пл., 4
\end{centering}
}


\end{document}